% Copyright (C) 2023 Alexander Rodin
% Licensed under Creative Commons Attribution 4.0 International Public License
% which is available at https://creativecommons.org/licenses/by-sa/4.0/legalcode

\documentclass[b5paper]{book}
\usepackage[T2A]{fontenc}
\usepackage[utf8]{inputenc}
\usepackage{amsmath,amssymb}
\usepackage[russian]{babel}
% \usepackage{opensans}
\usepackage[most]{tcolorbox}
\usepackage{titlesec}
\usepackage{indentfirst}

\titleformat{\subsection}
  {\normalfont\fontsize{10}{15}\bfseries}{\thesection}{1em}{}

\usepackage{mathtools}
\DeclarePairedDelimiter\bra{\langle}{\rvert}
\DeclarePairedDelimiter\ket{\lvert}{\rangle}
\DeclarePairedDelimiterX\braket[2]{\langle}{\rangle}{#1 \delimsize\vert #2}

\binoppenalty=10000
\relpenalty=10000

\definecolor{mathadd-bg}{RGB}{255, 241, 224}
% \definecolor{mathadd-bg}{RGB}{255, 232, 224}
\definecolor{mathadd-fg}{RGB}{173, 54, 33}
\newtcolorbox{mathadd}[2][]{%
    colback=mathadd-bg,
    grow to right by=-0.5mm,
    grow to left by=-0.50mm,
    boxrule=0pt,
    boxsep=0pt,
    breakable,
    enhanced jigsaw,
    borderline west={4pt}{0pt}{mathadd-fg},
%     title={#2\par},
    colbacktitle={mathadd-bg},
    coltitle={black},
    fonttitle={\large\bfseries},
    attach title to upper={},
    before upper={\parindent4pt},
    #1,
}

\begin{document}

\section*{Атом водорода}

Рассмотрим систему из двух заряженных частиц: протона и электрона. Такая система называется атомом водорода.
Классический гамильтониан для такой системы будет иметь вид

$$
H = \frac{p_e^2}{2m_e} + \frac{p_p^2}{2m_p} + \frac{e^2}{|r_e - r_p|}.
$$

Будем считать, что $m_e \ll m_p,$ тогда можно считать протон неподвижным и рассматривать только движение электрона. В таком случае
гамильтониан принимает вид

$$
H = \frac{p^2}{2m} + \frac{e^2}{r}.
$$

Будем рассматривать систему в сферических координатах, где

\begin{equation*}
\begin{aligned}
x &= r \sin \theta \cos \phi, \\
y &= r \sin \theta \sin \phi, \\
z &= r \cos \phi,
\end{aligned}
\end{equation*}

Тогда оператор импульса $\hat p,$ который при квантовании переходит в $-i\hbar \nabla,$ выразится как

$$
\hat p = -i\hbar \left({\mathbf e}_r\frac{\partial}{\partial r} + {\mathbf e}_{\theta} \frac 1 r \frac {\partial}{\partial \theta} + 
{\mathbf e}_{\phi}\frac{1}{r\sin\theta} \frac{\partial}{\partial \varphi}.
 \right)
$$

Таким образом, квантовомеханический гамильтониан в координатном представлении имеет вид

\begin{equation*}
\begin{aligned}
\hat H  &= \frac{\hat p^2}{2m} + \frac{e^2}{r} = \\
&= \frac{1}{2m} \left[\frac{1}{r^2}\frac{\partial}{\partial r}\left(r^2 \frac {\partial}{\partial r}\right)
 + \frac{1}{r^2 \sin\theta}\frac{\partial}{\partial \theta}
\left(\sin \theta \frac{\partial}{\partial \theta}\right)
+ \frac{1}{r^2\sin^2 \theta}\frac{\partial^2}{\partial \varphi^2}
\right] + \frac{e^2}{r}.
\end{aligned}
\end{equation*}

Соответствующее этому гамильтониану стационарное уравнение Шрёдингера:

\begin{equation}
\frac{1}{2m} \left[\frac{1}{r^2}\frac{\partial}{\partial r}\left(r^2 \frac {\partial\psi}{\partial r}\right)
 + \frac{1}{r^2 \sin\theta}\frac{\partial\psi}{\partial \theta}
\left(\sin \theta \frac{\partial}{\partial \theta}\right)
+ \frac{1}{r^2\sin^2 \theta}\frac{\partial^2 \psi}{\partial \varphi^2}
\right] + \frac{e^2}{r} \psi = E\psi.
\end{equation}

\end{document}