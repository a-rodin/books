% Copyright (C) 2023 Alexander Rodin

\documentclass[b5paper]{book}
\usepackage[T2A]{fontenc}
\usepackage[utf8]{inputenc}
\usepackage{amsmath,amssymb}
\usepackage[russian]{babel}
\usepackage{opensans}
\usepackage[most]{tcolorbox}
\usepackage{titlesec}
\usepackage{indentfirst}

\titleformat{\subsection}
  {\normalfont\fontsize{10}{15}\bfseries}{\thesection}{1em}{}

\usepackage{mathtools}
\DeclarePairedDelimiter\bra{\langle}{\rvert}
\DeclarePairedDelimiter\ket{\lvert}{\rangle}
\DeclarePairedDelimiterX\braket[2]{\langle}{\rangle}{#1 \delimsize\vert #2}

\binoppenalty=10000
\relpenalty=10000

\definecolor{mathadd-bg}{RGB}{255, 241, 224}
% \definecolor{mathadd-bg}{RGB}{255, 232, 224}
\definecolor{mathadd-fg}{RGB}{173, 54, 33}
\newtcolorbox{mathadd}[2][]{%
    colback=mathadd-bg,
    grow to right by=-0.5mm,
    grow to left by=-0.50mm,
    boxrule=0pt,
    boxsep=0pt,
    breakable,
    enhanced jigsaw,
    borderline west={4pt}{0pt}{mathadd-fg},
%     title={#2\par},
    colbacktitle={mathadd-bg},
    coltitle={black},
    fonttitle={\large\bfseries},
    attach title to upper={},
    before upper={\parindent4pt},
    #1,
}

\begin{document}

\section* {Одномерный гармонический осциллятор}

Рассмотрим одномерную механическую систему с классическим гамильтонианом
$$H(p, x) = \frac {p^2} {2m} + U(x)$$
в окрестности точки устойчивого равновесия $x = 0.$ Будем считать, что потенциальную энергию можно разложить в окрестности этой точки как
$$U(x) \approx U(0) + \frac 1 2 U''(0) x^2.$$
Предположив, что величина $U''(0)$ положительна, обозначим частоту малых колебаний как $\omega = \sqrt {\frac {U''(0)} {m} },$ в результате чего гамильтониан запишется в виде
$$H(p, x) = \frac {p^2} {2m} + \frac {m\omega^2} {2} x^2 + U(0).$$

Выбрав начало отсчета потенциальной энергии таким образом, что $U(0) = 0$ и проквантовав систему,
получим следующий квантовомеханический гамильтониан:
\begin{equation}\label{osc-hamiltonian}
 \hat H = \frac {\hat p^2} {2m} + \frac {m\omega^2} {2} \hat x^2.
\end{equation}

Найдем его уровни энергии и собственные функции. Переходить в координатное представление и непосредственно решать уравнение Шрёдингера достаточно сложно, для этого необходимо использовать метод Лапласа.

\subsection*{Операторы рождения и уничтожения}

Поступим другим путем. Начнем с того, что для удобства дальнейших вычислений выделим множитель $\hbar \omega$ в \eqref{osc-hamiltonian}:

\begin{equation}\label{osc-hamiltonian-hw}
 \hat H = \hbar \omega \left (\frac {\hat p^2} {2m\hbar\omega} + \frac {m\omega} {2\hbar} \hat x^2\right).
\end{equation}

Теперь попробуем <<факторизовать>> операторное выражение в скобках. Введем \textit{оператор уничтожения}
$$\hat a_{-} = \sqrt{\frac{m\omega}{2\hbar}} \hat x + i \frac {\hat p} {\sqrt{2 m \hbar \omega}}$$
и сопряженный ему \textit{оператор рождения}
$$\hat a_{+} = (\hat a_{-})^\dagger =  \sqrt{\frac{m\omega}{2\hbar}} \hat x - i \frac {\hat p} {\sqrt{2 m \hbar \omega}}.$$
Произведение $\hat a_{+} \hat a_{-}$ равно
\begin{equation}\label{osc-a-minis-a-plus}
 \hat a_{+} \hat a_{-} = \frac {\hat p^2} {2 m \hbar \omega} + \frac {m\omega} {2\hbar} \hat x^2 +
\frac {i} {2\hbar} [\hat x, \hat p].
\end{equation}
Коммутатор $[\hat x, \hat p]$ равен $i \hbar,$ так что из \eqref{osc-a-minis-a-plus} следует, что
\begin{equation}\label{osc-p-plus-x}
\frac {\hat p^2} {2 m \hbar \omega} + \frac {m\omega} {2\hbar} \hat x^2 = \hat a_{+} \hat a_{-} +
\frac {1} {2}.
\end{equation}
Подставляя \eqref{osc-p-plus-x} в \eqref{osc-hamiltonian-hw}, получаем
\begin{equation}\label{osc-hamiltonian-a-plus-a-minis}
 \hat H = \hbar \omega \left(\hat a_{+} \hat a_{-} + \frac {1} {2}\right).
\end{equation}

\subsubsection*{Основное состояние}

Из выражения \eqref{osc-hamiltonian-a-plus-a-minis} видно, что энергия ограничена снизу величиной
$\hbar \omega / 2$
в силу того, что $\hat a_{+} = (\hat a_{-})^\dagger$ и следующего общего свойства:

\begin{mathadd}{}
    Любой оператор вида $\hat A^{\dagger} \hat A$ является неотрицательно определенным, так как
    для всякого вектора $\ket{\psi}$
    $$\bra{\psi} \hat A^{\dagger} \hat A \ket{\psi} = (\hat A \ket{\psi})^{\dagger} (\hat A \ket{\psi}) \geq 0.$$

    Отсюда следует неотрицательность всех собственных значений такого оператора: если $\ket{\psi}$ -- нормированный собственный вектор $\hat A^{\dagger} \hat A$ с собственным значением $\lambda,$ то
    $$\lambda = \bra{\psi} \lambda \ket{\psi} = \bra{\psi} \hat A^{\dagger} \hat A \ket{\psi} \geq 0.$$
\end{mathadd}

Покажем, что энергия основного состояния совпадает с минимально возможной энергией:
$$E_0 = \frac {\hbar \omega} {2}.$$ Для этого найдем собственный вектор гамильтониана $\ket{0},$
такой, что $\hat H \ket{0} = E_0 \ket{0}.$ Таким вектором является вектор, построенный как решение
уравнения
$$\hat a_{-} \ket {0} = 0.$$

В координатном представлении это уравнение записывается как
$$\left[\sqrt{\frac{m\omega}{2\hbar}} x + \frac{i}{\sqrt{2m \hbar \omega}}
\left(-i \hbar \frac {d}{dx} \right)
\right] \psi_0(x) = 0.$$
Данное дифференциальное уравнение решается методом разделения переменных и имеет решение
$$\psi_0(x) = C e^{-\frac{m\omega}{\hbar}\frac{x^2}{2}}.$$
Эту функцию можно нормировать, вычислив гауссов интеграл
$$\int \psi_0^*(x) \psi_0(x) dx = \int |C|^2 e^{-\frac{m\omega}{\hbar}\frac{x^2}{2}} dx=
|C|^2 \sqrt{\frac{\pi \hbar}{m \omega}},$$
откуда для волновой функции основного состояния получаем выражение
\begin{equation}\label{osc-ground-state}
 \psi_0(x) = \left(\frac{m\omega}{\pi \hbar}\right)^{\frac 1 4} e^{-\frac{m\omega}{\hbar} \frac{x^2}{2}}.
\end{equation}

Нормированность волновой функции доказывает, что найденный вектор $\ket{0}$ является физическим
состоянием и энергия $E_0 = \hbar \omega / 2$ является низшим уровнем энергии.

\subsubsection*{Повышение и понижение энергии}

Пусть $\ket{E}$ -- состояние с энергией $E:$
$$\hat H \ket{E} = E \ket{E}.$$
Вычислим коммутатор операторов уничтожения и рождения:
$$[\hat a_{-}, \hat a_{+}] = \hat a_{-} \hat a_{+} - \hat a_{+} \hat a_{-} = \frac {i} {\hbar} [\hat p, \hat x] = 1.$$
Используя значение этого коммутатора, найдем собственное значение гамильтониана, соответствующее вектору
$\hat a_{+} \ket {E}:$
\begin{equation}
\begin{aligned}
 \hat H \hat a_{+} \ket{E} &=
 \hbar\omega\left(\hat a_{+} \hat a_{-} + \frac 1 2\right) \hat a_{+} \ket{E} =\\
 &= \hbar\omega \left[\hat a_{+} \left(\hat a_{+} \hat a_{-} + [\hat a_{-}, \hat a_{+}]\right)
    + \frac 1 2 \hat a_{+}\right] \ket{E} =\\
 &= \hat a_{+} \left[\hbar\omega \left(\hat a_{+} \hat a_{-} + \frac 1 2\right) + \hbar \omega\right] \ket{E} =\\
 &= \hat a_{+} \left(\hat H + \hbar \omega\right) \ket{E} =\\
 &= \hat a_{+} \left(E + \hbar \omega\right) \ket{E} =\\
 &= (E + \hbar\omega) \hat a_{+} \ket{E}.
\end{aligned}
\end{equation}

Аналогично можно показать, что вектору $\hat a_{-} E$ соответствует собственное значение $E - \hbar \omega.$

Таким образом, операторы $\hat a_{+}$ и $\hat a_{-}$ повышают и понижают соответственно
энергию состояния с заданной энергией на величину $\hbar \omega.$ Говорят, что они \textit{рождают}
и \textit{уничтожают} кванты энергии, откуда и появилось название этих операторов.

\end{document}
