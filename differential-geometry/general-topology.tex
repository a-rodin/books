% Copyright (C) 2023 Alexander Rodin

\documentclass[b5paper]{book}
\usepackage[T2A]{fontenc}
\usepackage[utf8]{inputenc}
\usepackage{amsmath,amssymb,amsthm}
\usepackage[russian]{babel}
\usepackage{opensans}
\usepackage[most]{tcolorbox}
\usepackage{titlesec}
\usepackage{indentfirst}
\usepackage{makeidx}
\usepackage[inline]{enumitem}
\usepackage{textcomp}

\titleformat{\subsection}
  {\normalfont\fontsize{10}{15}\bfseries}{\thesection}{1em}{}

\theoremstyle{definition}
\newtheorem{definition}{Определение}
\let\oldemptyset\emptyset
\let\emptyset\varnothing

\binoppenalty=10000
\relpenalty=10000

\makeindex

\begin{document}

\section* {Общая топология}

\subsubsection*{Топологические пространства}

\begin{definition}
 Пусть $X$ -- множество, а $\tau$ -- некоторое множество подмножеств $X.$ Тогда пару
 $(X, \tau)$ называют \index{Топологическое пространство} \emph{топологическим пространством}, если:
 \begin{enumerate}[label={\arabic*\textdegree.}]
  \item $X\in \tau,$ $\emptyset \in \tau;$
  \item для всякого (конечного или бесконечного) набора множеств
  $\{U_\alpha \in \tau: \alpha \in A\},$ где $A$ -- некоторое множество индексов,
  их объединение принадлежит $\tau:$ $\bigcup\limits_{\alpha \in A} U_\alpha \in \tau;$
  \item для всякого конечного набора множеств из $\tau:$
  $\{U_\alpha\in\tau: \alpha = 1, \dots , N\}$ их пересечение принадлежит $\tau:$
  $\bigcap\limits_{\alpha=1}^{N} U_\alpha \in \tau.$
 \end{enumerate}
\end{definition}
Множества, принадлежащие $\tau,$ называются \index{Открытое множество}
\emph{открытыми}, а само множество $\tau$ называется \index{Топология топологического пространства} \emph{топологией} топологического пространства $(X, \tau)$. В случаях, когда это не вызывает недоразумений,
топологическое пространство $(X, \tau)$ обозначают кратко как $X.$

% \begin{definition}
%
% \end{definition}

\printindex

\end{document}
